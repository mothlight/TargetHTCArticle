%% 
%% Copyright 2007, 2008, 2009 Elsevier Ltd
%% 
%% This file is part of the 'Elsarticle Bundle'.
%% ---------------------------------------------
%% 
%% It may be distributed under the conditions of the LaTeX Project Public
%% License, either version 1.2 of this license or (at your option) any
%% later version.  The latest version of this license is in
%%    http://www.latex-project.org/lppl.txt
%% and version 1.2 or later is part of all distributions of LaTeX
%% version 1999/12/01 or later.
%% 
%% The list of all files belonging to the 'Elsarticle Bundle' is
%% given in the file `manifest.txt'.
%% 
%% Template article for Elsevier's document class `elsarticle'
%% with harvard style bibliographic references
%% SP 2008/03/01

%\documentclass[preprint,12pt,authoryear]{elsarticle}  %default in the template
%\documentclass[preprint,10pt,authoryear]{elsarticle}

%% Use the option review to obtain double line spacing
%% \documentclass[authoryear,preprint,review,12pt]{elsarticle}

%% Use the options 1p,twocolumn; 3p; 3p,twocolumn; 5p; or 5p,twocolumn
%% for a journal layout:
%% \documentclass[final,1p,times,authoryear]{elsarticle}
%% \documentclass[final,1p,times,twocolumn,authoryear]{elsarticle}
 \documentclass[final,3p,times,authoryear]{elsarticle}
%% \documentclass[final,3p,times,twocolumn,authoryear]{elsarticle}
%% \documentclass[final,5p,times,authoryear]{elsarticle}
%% \documentclass[final,5p,times,twocolumn,authoryear]{elsarticle}

%% For including figures, graphicx.sty has been loaded in
%% elsarticle.cls. If you prefer to use the old commands
%% please give \usepackage{epsfig}

%% The amssymb package provides various useful mathematical symbols
\usepackage{amssymb}
%% The amsthm package provides extended theorem environments
 \usepackage{amsthm}
 \usepackage{amsmath}
 \usepackage{color}
 \usepackage{amsmath}
\usepackage{siunitx}


\usepackage{framed} % Framing content
\usepackage{multicol} % Multiple columns environment
\usepackage{nomencl} % Nomenclature package
\makenomenclature
%\setlength{\nomitemsep}{-\parskip} % Baseline skip between items
\setlength{\nomitemsep}{0.01cm}
\renewcommand*\nompreamble{\begin{multicols}{2}}
\renewcommand*\nompostamble{\end{multicols}}
\newcommand{\degreeC}{\ensuremath{^\circ}C }

\usepackage[nonumberlist]{glossaries}
\makeglossaries 


%% The lineno packages adds line numbers. Start line numbering with
%% \begin{linenumbers}, end it with \end{linenumbers}. Or switch it on
%% for the whole article with \linenumbers.
%% \usepackage{lineno}

\journal{Urban Climate}


\begin{document}

\newglossaryentry{Tmrt}{name={$T_{mrt}$},symbol={\ensuremath{T_{mrt}}},description={mean radiant temperature (\SI{}{\degreeCelsius}}}
\newglossaryentry{UTCI}{name={$UTCI$},symbol={\ensuremath{UTCI}},description={universal thermal climate index}}
\newglossaryentry{Tsurf}{name={$T_{surf}$},symbol={\ensuremath{T_{surf}}},description={surface temperature from the force-restore model (\degreeC)}}
\newglossaryentry{Tg}{name={\ensuremath{T_{g}}},symbol={\ensuremath{T_{g}}},description={globe temperature (K)}}
\newglossaryentry{Tcan}{name={\ensuremath{T_{can}}},symbol={\ensuremath{T_{can}}},description={canopy air temperature (z < z$_{H}$) (K)}} 
\newglossaryentry{A}{name={\ensuremath{A}},symbol={\ensuremath{A}},description={surface area of a sphere of diameter D, of value 0.15m, (=$\pi 0.15^{2}$m$^{2}$)}}
\newglossaryentry{D}{name={\ensuremath{D}},symbol={\ensuremath{D}},description={diameter of sphere (m)}}
\newglossaryentry{Nu}{name={\ensuremath{Nu}},symbol={\ensuremath{Nu}},description={Nusselt number (-)}}
\newglossaryentry{Re}{name={\ensuremath{Re}},symbol={\ensuremath{Re}},description={Reynolds number (-)}}
\newglossaryentry{Pr}{name={\ensuremath{Pr}},symbol={\ensuremath{Pr}},description={Prandtl number (-)}}
\newglossaryentry{k}{name={\ensuremath{k}},symbol={\ensuremath{k}},description={thermal conductivity of the fluid (i.e. air) (W m$^{-1}$ K$^{-1}$)}}


\begin{frontmatter}

%% Title, authors and addresses

%% use the tnoteref command within \title for footnotes;
%% use the tnotetext command for theassociated footnote;
%% use the fnref command within \author or \address for footnotes;
%% use the fntext command for theassociated footnote;
%% use the corref command within \author for corresponding author footnotes;
%% use the cortext command for theassociated footnote;
%% use the ead command for the email address,
%% and the form \ead[url] for the home page:
%% \title{Title\tnoteref{label1}}
%% \tnotetext[label1]{}
%% \author{Name\corref{cor1}\fnref{label2}}
%% \ead{email address}
%% \ead[url]{home page}
%% \fntext[label2]{}
%% \cortext[cor1]{}
%% \address{Address\fnref{label3}}
%% \fntext[label3]{}

\title{Target and HTC}


%% use optional labels to link authors explicitly to addresses:


\author[melb,monash,crc]{Kerry~A.~Nice\corref{cor1}}
\ead{kerry.nice@unimelb.edu.au}
\author[az1,az2,monash,crc]{Ashley Broadbent}
\author[monash,crc]{Stephanie Jacobs}


\cortext[cor1]{Principal corresponding author}
\address[melb]{Transport, Health, and Urban Design Hub, Faculty of Architecture, Building, and Planning, University of Melbourne, Victoria 3010, Australia}
\address[monash]{School of Earth, Atmosphere and Environment, Monash University, Clayton, VIC 3800, Australia}
\address[az1]{School of Geographical Sciences and Urban Planning, Arizona State University, Tempe, Arizona, USA}
\address[az2]{Urban Climate Research Center, Arizona State University, Tempe, Arizona, USA}
\address[crc]{Cooperative Research Centre for Water Sensitive Cities, Melbourne, Australia}





\begin{abstract}



\end{abstract}

\begin{keyword}
micro-climate modelling, urban vegetation, human thermal comfort
%% keywords here, in the form: keyword \sep keyword

%% PACS codes here, in the form: \PACS code \sep code

%% MSC codes here, in the form: \MSC code \sep code
%% or \MSC[2008] code \sep code (2000 is the default)

\end{keyword}

\end{frontmatter}







\section{Introduction}\label{sec:introduction}





\section{Model Overview}\label{sec:ModelOverview}



\subsection{Mean radiant temperature and Universal Thermal Climate Index calculations}\label{sec:tmrtutci}

VTUF-3D provides output of incoming and outgoing shortwave and longwave radiation and \glssymbol{Tsurf} for each facet (surface) at each timestep. Air temperature for the canyon, \glssymbol{Tcan}, is also provided for each timestep. Using these and values of vapour pressure from the forcing data and wind speed at street level, the values for mean radiant temperature (\glssymbol{Tmrt}) and the human thermal comfort index UTCI are calculated for each surface. User defined options in the model are available to either calculate $L\downarrow$, using the clear sky formula of \cite{Prata1996}, or using forcing values. Based on this, different options will be used to calculate the following equations.

 

Calculations of \glssymbol{Tmrt} (\SI{}{\degreeCelsius}) uses a two step procedure. First, globe temperature (\glssymbol{Tg}) is calculated by an iterative relaxation solution, using a formulation of \cite{Liljegren2008} in  

\begin{equation}\label{eq:tg2}
\begin{split}
\glssymbol{A}\epsilon_{g}\sigma T_{g}^{4} &= \frac{\glssymbol{A}}{2} \epsilon_{g}\sigma( \epsilon_{a} T_{a}^{4} +  \epsilon_{sfc} T_{sfc}^{4} ) \\
&+ \frac{\glssymbol{A}}{2}( 1-\alpha_{g})(1-f_{dir})S  \\
&+ \frac{\glssymbol{A}}{4}( 1-\alpha_{g})f_{dir}S /\cos(\theta) \\
&+ \frac{\glssymbol{A}}{2}( 1-\alpha_{g})\alpha_{sfc}S \\
&- \glssymbol{A}h(T_{g}-T_{a})   
\end{split}
\end{equation}



where \glssymbol{A}, is the \glsdesc{A},
$\epsilon_{a}$, the longwave emissivity of the atmosphere, 
$\epsilon_{sfc}$, the longwave emissivity of the surface, 
$\epsilon_{g}$, the globe emissivity (of value 0.95), 
$h$, the convective heat transfer coefficient (W m$^{-2}$ K$^{-1}$) (see Equation (\ref{eq:h})), 
$T_{a}$, the dry bulb air temperature (K) (using $T_{can}$), 
$T_{sfc}$, the surface temperature (K), 
S, the horizontal solar irradiance (W m$^{-2}$) calculated from the total of absorbed and reflected shortwave, 
$\theta$, the solar zenith angle, 
$\alpha_{sfc}$, the surface albedo (of value 0.15),  
$\alpha_{g}$, the globe albedo (of value 0.05), and 
$f_{dir}$, the fraction of the total horizontal solar irradiance, 
$S$, due to the direct beam of the sun. 




The convective heat transfer coefficient, $h$, as used in Equation (\ref{eq:tg2}), is calculated using 

\begin{equation}\label{eq:h}
Nu = 2.0 + 0.6Re^{1/2}Pr^{1/3};  ~~h = k / D Nu
\end{equation}

where $Nu$ is the Nusselt number (-),
$Re$, the Reynolds number (-),
$Pr$, the Prandtl number (-), and 
$k$, the thermal conductivity of the fluid (i.e. air) (W m$^{-1}$K$^{-1}$) \citep{Liljegren2008}.



Depending on user defined scenario model configuration settings, a number of terms in Equation (\ref{eq:tg2}) can use internally calculated values. $\sigma (\epsilon_{a} T_{a}^{4} + \epsilon_{sfc} T_{sfc}^{4} )$ can be replaced with $L\downarrow + L\uparrow$, the result of reflections of $L \downarrow_{i,sky}$. $(1-f_{dir})S$ can be replaced with $K \downarrow_{dif}$, and $f_{dir}S/ \cos(\theta)$ with $K \downarrow_{dir}$. Also, for vegetated grid squares, the terms for $L\downarrow$ and $L\uparrow$ can be replaced with calculations of $L\downarrow$ and $L\uparrow$ from $\epsilon \sigma T^{4}$ where $T_{s,1}$, soil surface temperature (K), is used as the temperature term for the $L\uparrow$ calculation while leaf temperature is used for $L\downarrow$. 

Using $T_{g}$, calculated in Equation (\ref{eq:tg2}), the second step in calculating $T_{mrt}$ proceeds in a formulation of \cite{Kantor2011} 
\begin{equation}\label{eq:tmrtbucket}
  T_{mrt} = 
  \bigg(
   (T_{g}+273.15)^{4} + 
    \frac{1.1 \times 10^{8}  ws_{cm}^{0.6}}{\epsilon_{g}  D^{0.4}}
    \times 
     (T_{g}-T_{a})
    \bigg)^{0.25} - 273.15
\end{equation}
 where $ws_{cm}$ is the wind speed (cm s$^{-1}$).




Finally, the Universal Thermal Climate Index (UTCI) (\SI{}{\degreeCelsius}) for each surface is calculated using the \cite{Brode2009u} UTCI formula, generating UTCI values for each surface, allowing human thermal comfort to be examined in detail across a modelled domain. The calculation uses modelled outputs of air temperature, wind speed (using model calculated wind speed at canyon level), relative humidity (from the forcing data), and $T_{mrt}$ (calculated above). VTUF-3D does not currently calculate a canopy level relative humidity. This is a current limitation and a future enhancement for the model. But as will be shown in the evaluation of VTUF-3D focusing on urban canopy layer air temperature, $T_{mrt}$, and UTCI predictions (Section \ref{sec:CoMValidations}) and in the energy flux evaluation (Section \ref{sec:PrestonValidation}), the model still performs well despite this limitation. 




\subsection{Data inputs}\label{sec:datainputs}
\subsubsection{Land cover}\label{sec:landcover}

%The model uses simple data inputs, which are designed to be easily accessible. The model requires the user to define the land cover fractions of buildings (\glssymbol{Fbldg}), concrete surfaces (\glssymbol{Fconc}), asphalt surfaces (\glssymbol{Fasph}), grass (\glssymbol{Fgras}), irrigated grass (\glssymbol{Figrs}), tree (\glssymbol{Ftree}), and water (\glssymbol{Fwatr}). These land cover categories are self-explanatory and describe most of the surfaces in urban areas. Low vegetation (shrubs and bushes) can be included in \glssymbol{Ftree} and \glssymbol{Fbldg} collectively refers to all buildings in urban areas.  In addition to the land cover fractions, the model requires average building height (\glssymbol{BH}) and street width (\glssymbol{W}) information. These data are only needed if the user intends to calculate \glssymbol{Tac}­. In future, look up tables for these land cover variables, based on land-use or local-climate zones (LCZ) \citep{Stewart2012a} could be developed. 

\subsubsection{Meteorological data}\label{sec:metdata}

%\glssymbol{Toolkit2} requires reference meteorological data to force the model. The model requires the following inputs: incoming shortwave radiation (\glssymbol{kdown}), incoming long wave radiation (\glssymbol{ldown}) (can be modelled if not available, see \ref{sec:app}), relative humidity (\glssymbol{RH}), reference level wind speed (\glssymbol{Uz}), and reference level air temperature (\glssymbol{Ta}).  Reference meteorological should be representative of background meteorological conditions (i.e. not significantly affected by microclimate effects).  We recommend using the nearest BoM airport weather station for reference meteorological data. BoM radiation data from a small selection of sites is freely available online\footnote{There are options available for modelling K↓, such as the Net All-wave Radiation Parameterization (NARP) of \cite{Offerle2003}, which could be integrated if needed.}.  However, other meteorological variables (\glssymbol{Ta}, \glssymbol{RH}, and \glssymbol{Uz}) are not always freely available and may need to be purchased from the BoM for a small fee. Alternatively, the CRCWSC could generate and make available reference meteorology datasets for each of Australian capital city. 

\section{Model description}\label{sec:ModelDescription}
\subsection{Net energy}\label{sec:net}
%
%During daylight hours (\glssymbol{kdown} $\geq$ 10) net radiation (\glssymbol{Rn}) for the ith surface type is calculated using the following \citep{Loridan2011}
%
%
%\begin{equation} 
%\glssymbol{Rn}  
% = \glssymbol{kdown} 
% (1-\glssymbol{albedo}_{i}) +  \glssymbol{epsilon}_{i}(\glssymbol{ldown}-\glssymbol{sigma} \glssymbol{Ta}^{4}) - 0.08\glssymbol{kdown} (1- \glssymbol{albedo}_{i})) 
%\label{eq:rn} \end{equation} 
%
%
%%Why is T_surf not directly taken from the model instead of T_a (also for the day)? Then you don't need a correction factor.
%%How is this correction factor determined? Is this correction factor site- and meteo- independent?
%
%% It is not clear to me why does calculation Rn involves  T_surf(t-2) and not T_surf(t-1). T_surf(t-1) is output from 'force-restore', so a subsequent call of the 'radiation balance' in the current timestep (t) would be able to use T_surf(t-1).
%
%where \glssymbol{albedo}$_{i}$ is \glsdesc{albedo}, \glssymbol{epsilon}$_{i}$ is \glsdesc{epsilon}, \glssymbol{Ta} is forcing air temperature (K), and \glssymbol{sigma} is the \glsdesc{sigma}. The \glssymbol{albedo}$_{i}$ and \glssymbol{epsilon}$_{i}$ values are predefined for each surface (see Table \ref{tab:Parameter}), validation report).  The right hand side of the equation accounts for net longwave radiation, and \glssymbol{lup} is approximated using \glssymbol{Ta}. \cite{Loridan2011}. include the term 0.08\glssymbol{kdown}$(1-\glssymbol{albedo}_{i})$ to account for the difference between near surface \glssymbol{Ta} and \glssymbol{Tsurf}. However, currently there is no correction for this difference at night. As such, we substitute Ta for. The modelled \glssymbol{Tsurf} (from 2 time steps (t) previous) is used to calculate L↑ at night. It is necessary to use  due to the method used to calculate the storage heat flux (eq. \ref{eq:ohm}), which requires \glssymbol{Rn} from the previous time step.  Testing suggests this lag does not significantly affect calculations when a 30 minute time step is used. Thus, at night (\glssymbol{kdown} $<$ 10) net radiation for i surface type is
%
%\begin{equation} 
%\glssymbol{Rn}  
% = \glssymbol{kdown} 
% (1-\glssymbol{albedo}_{i}) + \glssymbol{epsilon}_{i}] \big(\glssymbol{ldown}-\glssymbol{sigma} T_{surf[t-2]}^{4} \big) 
%\label{eq:rn2} \end{equation} 
%
%
%The \glssymbol{Rn} for each surface type is then used to calculate a surface energy balance for each surface. 
\subsection{Surface energy balance (LUMPS)}\label{sec:lumps}
%
%The energy balance calculations are based on the Local-scale Urban Meteorological Parameterisation Scheme (LUMPS) \citep{Grimmond2002a}. For non-water surfaces, the sensible (\glssymbol{H}) and latent (\glssymbol{LE}) heat fluxes for the $i$th surface type are calculated using
%
%
%\begin{equation} 
%\glssymbol{H}_{i} = 
%\frac{(1-\glssymbol{pm}_{i})+(\frac{\glssymbol{gamma}}{\glssymbol{s}})}{1+\frac{\glssymbol{gamma}}{\glssymbol{s}}}
%(R_{n,i} -\Delta S_{i})- \glssymbol{beta}
%\label{eq:Hi} \end{equation} 
%
%\begin{equation} 
%\glssymbol{LE}_{i} = 
%\frac{\glssymbol{pm}_{i}}{1+\frac{\glssymbol{gamma}}{\glssymbol{s}}}
%(R_{n,i}-\Delta S_{i})+ \glssymbol{beta}
%\label{eq:LE} \end{equation} 
%
%where \glssymbol{s} is the slope of the saturation vapour pressure-versus-temperature curve, \glssymbol{gamma} is the psychrometric constant and \glssymbol{pm}$_{i}$ and \glssymbol{beta} are based on a simplification of the Penman–Monteith approach, which takes into account the Priestley-Taylor coefficient; \glssymbol{pm}$_{i}$ depends on the surface moisture status, and \glssymbol{beta} is an empirical constant, set to 3 W m$^{-2}$ \citep{Grimmond2002a}. By default, the \glssymbol{pm} parameter is set using values from \cite{Hanna1992} (Table \ref{tab:surftype}). Alternatively, \glssymbol{pm} for vegetated classes can be defined as a function of stomatal resistance (sr) using $\glssymbol{pm} = 6.9\times10^{-6} sr^{2} - 0.004 sr + 1.3$ \citep{DeBruin1983}.  
%
%The objective hysteresis model (OHM) is used to calculate storage heat flux for the $i$th land cover class  \citep{Grimmond2002a} 
%
%\begin{equation} 
%\Delta S_{i} = R_{n,i} a_{1,i} + \Big( \frac{\partial R_{n,i}}{\partial t}   \Big)a_{2,i} + a_{3,i}
%\label{eq:ohm} \end{equation} 
%
%%So this means that the heat storage (hence the heat transfer between the urban canopy  atmosphere, as the residual) is parametrized according to downward radiation and the building parameters. This means that the dependency on the other atmospheric conditions, such as air temperature, wind speed (ventilation), and humidity, is neglected.
%
%%It is not clear to me how the presented a-parameters actually relate to the physical characteristics of the urban canopy (urban envelope, building/street heat capacity and conductivity, vegetation etc.). 
%
%% I think these things need to be better described/sustained in the technical description.  
%
%where the three coefficients \glssymbol{a1}, \glssymbol{a2}, and \glssymbol{a3}, are defined for each surface (see examples in Table \ref{tab:Parameter}), and $\frac{\partial R_{n,i}}{\partial t} =0.5(R_{n,it-1} - R_{n,it+1})$  .  The $\Delta S_{i}$ is then used to calculate the surface temperature using the `force-restore' method.
%
%
%\subsection{Surface temperature calculation (`force-restore')}\label{sec:tsurf}
%
%The change in surface temperature \glssymbol{Tsurf} for surface $i$ with respect to time (t) is calculated as
%
%%How do you get this kind of force-restore method? Could you provide a short explanation what it (physically) represents and a reference?
%
%\begin{equation} 
%\frac{\partial T_{surf,i}}{\partial t}= \frac{\Delta S_{i}}{C_{i} D} - \frac{2 \pi}{\tau} (T_{surf,i -1} - T_{m,it-1})
%\label{eq:force} \end{equation} 
%
%% period = day ??
%
%Where $C_{i}$ is the \glsdesc{C}, $\tau$ is the period (86400 seconds), $D = 2 \glssymbol{kappa} _{i}  / \omega 0.5$, $\glssymbol{kappa} _{i}$ is the \glsdesc{kappa}, and $\glssymbol{kappa} = 2\pi / \tau$.  The first term on the right-hand side is the forcing term that affects \glssymbol{Tsurf}, while the second term is the restore term which dampens the forcing term. The \glsdesc{Tm} ($T_{m,i}$) is calculated using
%
%%what are the units of the different variables? It would make it more easy to physically understand the different terms
%
%%It's not clear to me what this 'damping'-term means. Is this the heat transfer between the buildings and the ground below? Or is it to avoid numerical instability?
%
%\begin{equation} 
%\frac{\partial T_{m,i}}{\partial t} = \frac{\Delta S_{i}}{C_{i} D_{y}}
%\label{eq:tm} \end{equation} 
%
%where \glssymbol{Dy} = $D \sqrt{365}$, the \glsdesc{Dy}. 
%
%Thus for each time step, the aggregated \glssymbol{Tsurf} and \glssymbol{H} are equal to
%
%\begin{equation} 
%\glssymbol{H} = \sum_{i=1}^{6} (F_{i} H_{i}) + (F_{wtr} H_{wtr})
%\label{eq:H} \end{equation} 
%
%\begin{equation} 
%\glssymbol{Tsurf} = \sum_{i=1}^{6} (F_{i} T_{surf,i}) + (F_{wtr} T_{wtr})
%\label{eq:tsurf} \end{equation} 
%
%As the force-restore method cannot be applied to water, we use a simple a simple water body model to calculate $H_{wtr}$ and $T_{wtr}$.


\subsection{Simple water body model}\label{sec:simplewater}
%
%% what do you mean by water?
%
%The water model in the toolkit is based on a single water layer, overlaying a soil layer. Essentially, the force-restore surface temperature model is implemented, and is overlain by a homogeneous mixed water layer (i.e. neglecting thermal stratification) representing a water body of depth z(m). The model is designed to apply to water bodies of 0.1-1.0m depths. The water model is based on the pan evaporation model of \cite{MolinaMartinez2006} which closely follows that of the lake model of \cite{Jacobs1998}. The water body model also determines the surface energy balance of the water surface. The energy balance model for the water layer is given by \citep{MolinaMartinez2006}
%
%\begin{equation} 
%\glssymbol{Sab} + \glssymbol{l*} + H_{wtr} - LE_{wtr} - \glssymbol{Gwtr} -\glssymbol{deltaSwtr} = 0
%\label{eq:sab} \end{equation} 
%
%
%where \glssymbol{Sab} is \glsdesc{Sab}, \glssymbol{l*}, the \glsdesc{l*}, \glssymbol{Gwtr} is the \glsdesc{Gwtr}, and \glssymbol{deltaSwtr} is the \glsdesc{deltaSwtr}. Solar radiation penetrates the water surface and is absorbed as described by Beer's Law \citep{MolinaMartinez2006}
%
%\begin{equation} 
%\glssymbol{Sab} = \glssymbol{k*} \big[ \glssymbol{betak} + (1 - \glssymbol{betak}) )(1-e^{-\glssymbol{eta}})  \big]
%\label{eq:sab2} \end{equation} 
%
%where \glssymbol{k*} is the \glsdesc{k*}, \glssymbol{betak} is the \glsdesc{betak} \citep{MolinaMartinez2006}, and \glssymbol{eta} the \glsdesc{eta}. Here, \glssymbol{eta} is given the following from \cite{Subin2012a}, for the water layer with depth z(m)
%
%\begin{equation} 
%\glssymbol{eta} = 1.1925z^{-0.424}
%\label{eq:eta} \end{equation} 
%
%
%A correction factor for the solar path length zenith angle is often applied to Eq. \ref{eq:sab2} \citep{MolinaMartinez2006} but this has been omitted to reduce complexity. The convective heat transport \glssymbol{Gwtr} into the soil at the base of the water layer is given by \citep{MolinaMartinez2006}
%
%\begin{equation} 
%\glssymbol{Gwtr} = - \glssymbol{Cwtr} \glssymbol{kwtr} \frac{\Delta T}{\Delta z}
%\label{eq:gwtr} \end{equation} 
%
%
%where \glssymbol{Cwtr} is the \glsdesc{Cwtr}, \glssymbol{kwtr} is the \glsdesc{kwtr}, 
%and the change in depth $\Delta z = z$ (the depth of the water layer), the change in temperature $\Delta T$ (\degreeC) is the difference between the water temperature \glssymbol{Twtr} (\degreeC) and the soil temperature \glssymbol{Ts} (\degreeC). These initial temperatures (\glssymbol{Twtr} and \glssymbol{Ts}) must be specified (at time t = 0, they can be set to the same value). \glssymbol{kwtr} is a complex function accounting for thermal stratification and surface friction velocity. Again, to reduce complexity and assuming a mixed homogeneous water layer, a constant \glssymbol{kwtr} has been selected based on shallow lakes reported in \cite{SalasDeLeon2016} (Table \ref{tab:eddydiff}).
%
%
%The latent heat flux ($LE_{wtr}$) (W m$^{-2}$) is given by \citep{Arya2001}
%
%\begin{equation} 
%LE_{wtr} = \glssymbol{rhov} \glssymbol{Lv} \glssymbol{hv} \glssymbol{Uz} (\glssymbol{qs} - \glssymbol{qa})
%\label{eq:lewtr} \end{equation} 
%
%
%where \glssymbol{rhov} is the \glsdesc{rhov}, \glssymbol{Lv} is the \glsdesc{Lv}, \glssymbol{hv} is the \glssymbol{hv} \citep{Jones2005,Hicks1972}, \glssymbol{Uz}, the BoM reference wind speed, \glssymbol{qs} the saturated specific humidity at $T_{wtr}$, and \glssymbol{qa} is the specific humidity of the air at \glssymbol{Ta} (see appendix for calculation of \glssymbol{rhov}, \glssymbol{qs} and \glssymbol{qa}). The sensible heat flux is given by \cite{MolinaMartinez2006}\footnote{The temperature gradient (\glssymbol{Ta}-\glssymbol{Ts}) in Eq. 6 from \cite{MolinaMartinez2006} pg. 254 is reversed here from the usual (\glssymbol{Ts}-\glssymbol{Ta}). It accounts for the addition of sensible heat to the water layer from the energy balance equation (Eq. \ref{eq:rn}).}
%
%\begin{equation} 
%H_{wtr} = \glssymbol{rhoa} \glssymbol{cp} \glssymbol{hc} \glssymbol{Uz} (\glssymbol{Ta}-\glssymbol{Ts})
%\label{eq:hwtr} \end{equation} 
%
%%Here, a resistence formulation is used to calculated the sensible heat flux (hence depending on wind temperature etc.), whereas for the energy balance regarding the (non-water covered) buildings, QH is calculated as the residual (and not temperature and wind-speed dependent). Such different model formulation might perhaps lead to artificial non-physical discrepancies.
%
%where \glssymbol{rhoa} is the \glsdesc{rhoa}, \glssymbol{cp} the \glsdesc{cp}, and \glssymbol{hc} the \glsdesc{hc}. Returning to Eq. \ref{eq:sab2}, net long wave radiation \glssymbol{l*} = \glssymbol{Rn} - \glssymbol{k*} leaving $\Delta S_{wrt}$ from the energy balance equation, which is defined as \citep{MolinaMartinez2006}
%
%\begin{equation} 
%\Delta S_{wrt} = \glssymbol{Cwtr} z \frac{\Delta T_{wtr}}{\Delta t}
%\label{eq:swrt} \end{equation}
%
%
%
%where $\Delta t$ is change in time (seconds) and \glssymbol{Cwtr} is the \glsdesc{Cwtr}. Solving for $\Delta T_{wtr}$ and adding the change in temperature to the previous time step ($T_{wtr} = T_{wtr,t-1}  + \Delta T_{wtr}$) gives the new water layer temperature. 
%
%
%Below the water layer, the Force-Restore model (Eq. \ref{eq:force}) determines the soil surface temperature where \glssymbol{Gwtr} is equivalent to $\Delta S_{i}$  for other surfaces. The soil layer parameters are set to saturated soil where the volumetric heat capacity is set to \glssymbol{C} $= 3.03 \times 10^{6}$ (J m$^{-2}$ K$^{-1}$) and the thermal diffusivity is set to \glssymbol{kappa} $= 0.625 \times 10^{-6}$ (m$^{2}$ s$^{-1}$). 
%
%%what about the the radiation that is not absorbed by the water body but hits the soil at the bottom (cfr. beer lambert law)? Perhaps this radiation-residual needs to be added to this Delta Si? 
%
%\subsection{Calculation of urban canopy layer air temperature (\glssymbol{Tac}) }\label{sec:calcTac}
%
%
%The \glssymbol{Tac} term is calculated following the approach outlined in Figure \ref{fig:Tac}. 
%
%
%\begin{figure}[!htbp]
%%\fbox{
%%\includegraphics[trim=13mm 43mm 32mm 55mm, clip,scale=0.60]{images/Tac.png}
%%}
% \caption{Schematic of the air temperature module.} \label{fig:Tac}
%\end{figure}
%
%
%First we, solve the following equation for \glssymbol{Tb} (Yang et al 2011 TODO FIND THIS REFERENCE). This assumes a generic local scale \glssymbol{Ta} for the site (all grid cells). 
%
%%What is meant by 'generic local scale'. I thought Ta is a temperature that is representative for the background environment (cfr. BoF) and not for the site itself.
%
%%What are the 'grid cells'?
%
%\begin{equation} 
%H = \glssymbol{rhoa} \glssymbol{cp} \frac{\glssymbol{Ta}-\glssymbol{Tb}}{\glssymbol{ra}}
%\label{eq:H_air} \end{equation}
%
%where \glssymbol{cp} is the \glsdesc{cp} and \glssymbol{rhoa} is the \glsdesc{rhoa}. 
%
%The canopy to atmosphere resistance (\glssymbol{ra}) is given by (Yang et al 2011 TODO FIND THIS REFERENCE):
%
%\begin{equation} 
%\glssymbol{ra} = \frac{ln^{2} \Bigg( \frac{\glssymbol{zu}-\glssymbol{d}}{\glssymbol{z0}}   \Bigg)   } {k^{2} \glssymbol{Uz}}
%\label{eq:ra} \end{equation}
%
%%It seems that a momentum-transfer resistance is calculated. Is it correct to used it in the above equation as a heat-transfer resistence? 
%
%Where \glssymbol{zu} is the \glsdesc{zu}, \glssymbol{d} is the \glsdesc{d}, and \glssymbol{z0} is the \glsdesc{z0}. 
%
%
%The temperature of the canopy layer (\glssymbol{Tac}) is then given by \citep{Oleson2010} 
%
%\begin{equation} 
%\glssymbol{Tac} = \frac{\glssymbol{Hca} + \glssymbol{Tb} + \glssymbol{Hcs} + \glssymbol{Tsurf} }  { \glssymbol{Hca}  + \glssymbol{Hcs} }
%\label{eq:tac} \end{equation}
%
%which is a simplification of Equation 3.75 (eq. \ref{eq:clmu375}) in the CLMU technical note \citep{Oleson2010}:, because we do not consider sensible heat conductance for individual surface types. Sensible heat conductance are given for the urban canopy layer to atmosphere (\glssymbol{Hca}) and surface to urban canopy layer (\glssymbol{Hcs}) by \citep{Oleson2010}
%
%
%
%\begin{equation} 
%\glssymbol{Hca} = \frac{1}{\glssymbol{ra}} \glssymbol{Hcs} = \frac{1}{\glssymbol{rs}}
%\label{eq:hca} \end{equation}
%
%where \glssymbol{rs} is the \glsdesc{rs}
%
%\begin{equation} 
%\glssymbol{rs} = \frac{\glssymbol{rhoa}\glssymbol{cp}}{11.8+4.2\glssymbol{Ucan}}
%\label{eq:rs} \end{equation}
% 				
%\glssymbol{Ucan} is the \glsdesc{Ucan}
%
%
%\begin{equation} 
%\glssymbol{Ucan} = \glssymbol{Utop} exp{ \Bigg( -0.386 \frac{\glssymbol{BH}}{\glssymbol{W}} \Bigg)}
%\label{eq:ucan} \end{equation}
%
%where \glssymbol{Utop} is the \glsdesc{Utop}, \glssymbol{BH} is the \glsdesc{BH} and \glssymbol{W} is the \glsdesc{W}. \glssymbol{Utop} is estimated at the top of the UCL based on \glssymbol{Uz} using a logarithmic relationship:
%
%
%
%
%\begin{equation} 
%\glssymbol{Utop} = \glssymbol{Uz} \frac{ ln \big( \frac{\glssymbol{BH}}{\glssymbol{z0}}  \big)}{ ln \big( \frac{\glssymbol{zu}}{\glssymbol{z0}}  \big) }
%\label{eq:xx4} \end{equation}
%
%If \glssymbol{BH} is below the domain average building height, it should be set to the domain average. 

\section{Model validation}\label{sec:validation}
\subsection{Overview}\label{sec:validationover}



%This paper provides a validation of the new microclimate model for the CRCWSC toolkit (hereafter referred to as \glssymbol{Toolkit2}). \glssymbol{Toolkit2} aims to calculate surface temperature (\glssymbol{Tsurf}) and surface level (urban canopy layer) air temperature (\glssymbol{Tac}) in urban areas using simple and well-known urban climate modelling methods. The model is primarily intended to calculate the cooling benefits of water sensitive urban design (WSUD) during hot and sunny summertime conditions. For a full description of \glssymbol{Toolkit2} see the technical document attached. For this validation we draw upon a range of data from Melbourne and Adelaide to test \glssymbol{Tsurf} and \glssymbol{Tac} calculations. 

\subsection{Methods and data}\label{sec:methods}
\subsubsection{Introduction}\label{sec:methodsintro}

%As part of the validation we conducted a range of simulations that test model performance for both \glssymbol{Tsurf} and \glssymbol{Tac}. These validation experiments are focussed on summertime conditions.  Firstly, we conducted detailed testing of model performance at simulating \glssymbol{Tsurf} for each land cover class that can be prescribed in \glssymbol{Toolkit2} (i.e. dry grass, asphalt etc.), using ground-based observations of \glssymbol{Tsurf} (section \ref{sec:landcoversim}). These simulations by land cover type, provide a detailed assessment of model parameters, and the underlying energy balance dynamics and resulting \glssymbol{Tsurf} for each land cover class. Secondly, we conducted suburb scale simulations of Mawson Lakes, Adelaide, for which we have high resolution remotely sensed \glssymbol{Tsurf} observations and in situ \glssymbol{Tac} data (section \ref{sec:suburbsim}).  The suburb scale simulations reflect the way the model is intended to be operationalised by practitioners.

\subsubsection{Land cover simulations}\label{sec:landcoversim} 

%To test model performance at simulating \glssymbol{Tsurf} of different land cover classes, and check a number of model parameters we utilised ground-based observations of \glssymbol{Tsurf} from the Melbourne metropolitan area. \cite{Coutts2016a} deployed infrared temperature sensors (SI-121 - Apogee), during February 2012, across a range of land cover types including: asphalt, concrete, grass, irrigated grass, steel roof, and water. The conditions during this period represented near-typical summertime conditions in Melbourne; including a number of days (15th, 24th, and 25th February) where air temperature exceeded 30\degreeC (see Figure \ref{fig:met}). These hotter days were characterised by northerly winds, which bring hot and dry air from Australia's interior, and often result in heatwave conditions in Melbourne. Additionally, there was at least one cloudy day where incoming shortwave radiation (\glssymbol{kdown}) dropped significantly and negligible amount of rainfall occurred (16th February).  In addition, to assess tree \glssymbol{Tsurf} we obtained a different set of observational data from the Monash cemetery tree experiment \citep{Coutts2016b}, including \glssymbol{Tsurf} observations of the tree canopy (collected during February 2014). We also obtained a BoM meteorological forcing data for the 2014 case study period.  This period (not shown) was very similar to the February 2012 period, with typical summer days punctuated by hotter and drier northerly conditions. 

\begin{figure}[!htbp]
%\fbox{
%\includegraphics[trim=0mm 0mm 0mm 0mm, clip,scale=0.65]{images/met1.png}
~
%\includegraphics[trim=0mm 0mm 0mm 0mm, clip,scale=0.65]{images/met2.png}
%}
 \caption{Meteorological conditions during land cover validation period. Data source: Melbourne Airport Bureau of Meteorology (ID 086282) weather station.} \label{fig:met}
\end{figure}

%
%The model was run for each of the 7 land cover types, and was forced with hourly data from Melbourne Airport Bureau of Meteorology (ID 086282) weather station data (Figure 2.1).  The following data were used as inputs: reference air temperature (\glssymbol{Ta}), incoming shortwave radiation (\glssymbol{kdown}), incoming long wave radiation (\glssymbol{ldown}), and relative humidity ($RH$). The model was spun up for 24 hours, although testing suggests minimal spin up is required. Initial conditions for average ground temperature (\glssymbol{Tm}) were obtained via Jacobs et al. (2000) TODO FIND REFERENCE and sensitivity testing. The model parameter setup is summarised in Table \ref{tab:Parameter}. As \glssymbol{Tac} was not calculated for these simulations we did not require wind speed or building height and street width information. 




\subsubsection{Suburb scale simulations (Mawson Lakes)}\label{sec:suburbsim} 
 

\subsection{Results}\label{sec:Results} 
\subsubsection{Land cover simulations}\label{sec:landcoverresult} 






































\section{Code availability}\label{sec:available}

\printglossary[title={List of Symbols}]

\section*{Acknowledgements}
The support of the Commonwealth of Australia through the Cooperative Research Centre program is acknowledged.
%\end{acknowledgements}

\section*{References}\label{sec:ref}
%% If you have bibdatabase file and want bibtex to generate the
%% bibitems, please use
%%
  \bibliographystyle{elsarticle-harv} 
  \bibliography{library}

%% else use the following coding to input the bibitems directly in the
%% TeX file.

\begin{thebibliography}{00}

%% \bibitem[Author(year)]{label}
%% Text of bibliographic item

\bibitem[ ()]{}

\end{thebibliography}


%% The Appendices part is started with the command \appendix;
%% appendix sections are then done as normal sections
\appendix
\setcounter{table}{0}
\renewcommand{\thetable}{A\arabic{table}}

%\subsection{}                               %% Appendix A1, A2, etc.


%%%%%%%%%% taking out parameterizations
\section{Appendix}\label{sec:app}  
\subsection{Additional data tables}\label{app:tables}  




%\authorcontribution{This work was developed by Kerry Nice and supervised by Andrew Coutts and Nigel Tapper. Model source code was received from Scott Krayenhoff and Remko Duursma (as acknowledged in Section \ref{sec:available}). Synthesis of this code and new code was developed by Kerry Nice. The article was written by Kerry Nice with editing and suggestions from Andrew Coutts and Nigel Tapper.}
%
%\begin{acknowledgements}
%The work described in this paper was developed during a PhD. project at Monash University. Funding for this was obtailed through the City of Melbourne, Monash University, and the CRC for Water Sensitive Cities.  
%\end{acknowledgements}

%\begin{acknowledgements}
%The support of the Commonwealth of Australia through the Cooperative Research Centre program is acknowledged.
%\end{acknowledgements}

%%%% \section{}
%%%% \label{}
%%\section{References}\label{sec:ref}
%%%% If you have bibdatabase file and want bibtex to generate the
%%%% bibitems, please use
%%%%
%%  \bibliographystyle{elsarticle-harv} 
%%  \bibliography{library}
%%
%%%% else use the following coding to input the bibitems directly in the
%%%% TeX file.
%%
%%\begin{thebibliography}{00}
%%
%%%% \bibitem[Author(year)]{label}
%%%% Text of bibliographic item
%%
%%\bibitem[ ()]{}
%%
%%\end{thebibliography}
%%
%%
%%
%%
%%



%
%
%%\nomenclature{$T_{mrt}$}{mean radiant temperature (\SI{}{\degreeCelsius})}  
%%\nomenclature{$UTCI$}{universal thermal climate index}
%
%\newglossaryentry{Tmrt}{name={$T_{mrt}$},symbol={\ensuremath{T_{mrt}}},description={mean radiant temperature (\SI{}{\degreeCelsius}}}
%\newglossaryentry{UTCI}{name={$UTCI$},symbol={\ensuremath{UTCI}},description={universal thermal climate index}}}
%
%\newglossaryentry{Tsurf}{name={$T_{surf}$},symbol={\ensuremath{T_{surf}}},description={surface temperature from the force-restore model (\degreeC)}}
%\newglossaryentry{Toolkit2}{name={Toolkit-2},symbol={Toolkit-2},description={CRC for Water Sensitive Cities micro-climate toolit model}}
%\newglossaryentry{Tac}{name={$T_{ac}$},symbol={\ensuremath{T_{ac}}},description={street level (urban canopy layer) air temperature (\degreeC)}}
%\newglossaryentry{Uz}{name={$U_{z}$},symbol={\ensuremath{U_{z}}},description={BoM reference wind speed (m s$^{-1}$)}}
%\newglossaryentry{Tb}{name={$T_{b}$},symbol={\ensuremath{T_{b}}},description={the air temperature above the urban canopy layer (\degreeC)}}
%\newglossaryentry{Fbldg}{name={$F_{bldg}$},symbol={$F_{bldg}$},description={land cover building fraction (\%)}} 
%\newglossaryentry{Fconc}{name={$F_{conc}$},symbol={$F_{conc}$},description={land cover concrete fraction (\%)}} 
%\newglossaryentry{Fasph}{name={$F_{asph}$},symbol={$F_{asph}$},description={land cover asphalt fraction (\%)}} 
%\newglossaryentry{Fgras}{name={$F_{gras}$},symbol={$F_{gras}$},description={land cover grass fraction (\%)}} 
%\newglossaryentry{Figrs}{name={$F_{igrs}$},symbol={$F_{igrs}$},description={land cover irrigated grass fraction (\%)}} 
%\newglossaryentry{Ftree}{name={$F_{tree}$},symbol={$F_{tree}$},description={land cover tree fraction (\%)}} 
%\newglossaryentry{Fwatr}{name={$F_{watr}$},symbol={$F_{watr}$},description={land cover water fraction (\%)}} 
%\newglossaryentry{W}{name={$W$},symbol={\ensuremath{W}},description={average street width (m)}}
%\newglossaryentry{BH}{name={$hH_{b}$},symbol={\ensuremath{h_{b}}},description={average building height (m)}}
%\newglossaryentry{kdown}{name={$K\downarrow$},symbol={\ensuremath{K\downarrow}},description={incoming shortwave radiation (W m$^{-2}$)}}
%\newglossaryentry{lup}{name={$L\uparrow$},symbol={\ensuremath{L\uparrow}},description={outgoing longwave radiation (W m$^{-2}$)}}
%\newglossaryentry{ldown}{name={$L\downarrow$},symbol={\ensuremath{L\downarrow}},description={incoming longwave radiation (W m$^{-2}$)}}
%\newglossaryentry{l*}{name={$L_{n}$},symbol={\ensuremath{L_{n}}},description={net longwave radiation (W m$^{-2}$)}}
%\newglossaryentry{k*}{name={$K_{n}$},symbol={\ensuremath{K_{n}}},description={net shortwave radiation (W m$^{-2}$)}}
%\newglossaryentry{RH}{name={$RH$},symbol={\ensuremath{RN}},description={relative humidity (\%)}}
%\newglossaryentry{Ta}{name={$T_{a}$},symbol={\ensuremath{T_{a}}},description={air temperature from the nearest BoM site (2 m) (\degreeC)}}
%\newglossaryentry{Rn}{name={$R_{n}$},symbol={\ensuremath{R_{n}}},description={net radiation (W m$^{-2}$)}}
%\newglossaryentry{albedo}{name={$\alpha$},symbol={\ensuremath{\alpha}},description={surface albedo}}
%\newglossaryentry{epsilon}{name={$\epsilon$},symbol={\ensuremath{\epsilon}},description={surface emissivity}}
%\newglossaryentry{sigma}{name={$\sigma$},symbol={\ensuremath{\sigma}},description={Stefan-Boltzmann constant (=5.67 $\times$ 10$^{-8}$ W m$^{-2}$K$^{-4}$)}} 
%\newglossaryentry{H}{name={$H$},symbol={\ensuremath{H}},description={sensible heat flux from LUMPS (W m$^{-2}$)}}
%\newglossaryentry{LE}{name={$LE$},symbol={\ensuremath{LE}},description={latent heat flux from LUMPS (W m$^{-2}$)}}
%\newglossaryentry{pm}{name={$pm$},symbol={\ensuremath{pm}},description={LUMPS empirical parameter (`alpha’ parameter) - relates to surface moisture}}
%\newglossaryentry{s}{name={$s$},symbol={\ensuremath{s}},description={slope of the saturation vapour pressure-versus-temperature curve }} 
%\newglossaryentry{gamma}{name={$\gamma$},symbol={\ensuremath{\gamma}},description={psychrometric constant}}
%\newglossaryentry{DeltaS}{name={$\Delta S$},symbol={\ensuremath{\Delta S}},description={storage heat flux from LUMPS (W m$^{-2}$)}}
%\newglossaryentry{beta}{name={$\beta$},symbol={\ensuremath{\beta}},description={LUMPS empirical parameter (beta parameter)}}
%\newglossaryentry{betak}{name={$\beta_{k}$},symbol={\ensuremath{\beta_{k}}},description={amount of shortwave radiation immediately absorbed by the water layer (set to 0.45)}}
%\newglossaryentry{a1}{name={$a_{1}$},symbol={\ensuremath{a_{1}}},description={Objective hysteresis model (OHM) parameter}}
%\newglossaryentry{a2}{name={$a_{2}$},symbol={\ensuremath{a_{2}}},description={Objective hysteresis model (OHM) parameter}}
%\newglossaryentry{a3}{name={$a_{3}$},symbol={\ensuremath{a_{3}}},description={Objective hysteresis model (OHM) parameter}}
%\newglossaryentry{C}{name={$C$},symbol={\ensuremath{C}},description={volumetric heat capacity (J m$^{-3}$ K$^{-1}$)}}
%\newglossaryentry{kappa}{name={$\kappa$},symbol={\ensuremath{\kappa}},description={thermal diffusivity (m$^{2}$ s$^{-1}$)}}
%\newglossaryentry{Tm}{name={$T_{m}$},symbol={\ensuremath{T_{m}}},description={average soil (ground) temperature (\degreeC)}}
%\newglossaryentry{Dy}{name={$D_{y}$},symbol={\ensuremath{D_{y}}},description={damping depth for the annual temperature cycle (m)}}
%\newglossaryentry{Sab}{name={$S_{ab}$},symbol={\ensuremath{S_{ab}}},description={absorbed shortwave radiation (W m$^{-2}$)}}
%\newglossaryentry{Gwtr}{name={$G_{wtr}$},symbol={\ensuremath{G_{wtr}}},description={convective heat flux at the bottom of the water layer (and into the soil below) (W m$^{-2}$)}}
%\newglossaryentry{deltaSwtr}{name={$\Delta S_{wtr}$},symbol={\ensuremath{\Delta S_{wtr}}},description={change in heat storage of the water layer (W m$^{-2}$)}}
%\newglossaryentry{eta}{name={$\eta$},symbol={\ensuremath{\eta}},description={extinction coefficient}}
%\newglossaryentry{Cwtr}{name={$C_{wrt}$},symbol={\ensuremath{C_{wrt}}},description={volumetric heat capacity of water (4.18$\times$10$^{6}$ J m$^{-3}$ K$^{-1}$)}}
%\newglossaryentry{kwtr}{name={$\kappa_{wtr}$},symbol={\ensuremath{\kappa_{wtr}}},description={eddy diffusivity of water (m$^{2}$ s$^{-1}$)}}
%\newglossaryentry{Twtr}{name={$T_{wtr}$},symbol={\ensuremath{T_{wtr}}},description={water surface temperature from simple water-body model (\degreeC)}}
%\newglossaryentry{Ts}{name={$T_{s}$},symbol={\ensuremath{T_{s}}},description={soil temperature (\degreeC)}}
%\newglossaryentry{rhov}{name={$\rho v$},symbol={\ensuremath{\rho v}},description={density of moist air (kg m$^{-3}$)}}
%\newglossaryentry{Lv}{name={$L_{v}$},symbol={\ensuremath{L_{v}}},description={latent heat of vaporisation (=2.43 MJ kg$^{-1}$)}}
%\newglossaryentry{hv}{name={$h_{v}$},symbol={\ensuremath{h_{v}}},description={bulk transfer coefficient for moisture (=1.4 $\times$ 10$^{-3}$)}}  
%\newglossaryentry{qa}{name={$q_{a}$},symbol={\ensuremath{q_{a}}},description={specific humidity (kg kg$^{-1}$)}}
%\newglossaryentry{qs}{name={$q_{s}$},symbol={\ensuremath{q_{s}}},description={saturated specific humidity (kg kg$^{-1}$) }}  
%\newglossaryentry{rhoa}{name={$\rho_{a}$},symbol={\ensuremath{\rho_{a}}},description={density of dry air (=1.2 kg m$^{-3}$)}} 
%\newglossaryentry{cp}{name={$c_{p}$},symbol={\ensuremath{c_{p}}},description={specific heat of air (1013 J kg$^{-1}$ K$^{-1}$)}}
%\newglossaryentry{hc}{name={$h_{c}$},symbol={\ensuremath{h_{c}}},description={bulk transfer coefficient for heat ($h_{c} = h_{v}$)}}
%%\newglossaryentry{Cwtr}{name={$C_{wtr}$},symbol={\ensuremath{C_{wtr}}},description={volumetric heat capacity of water (J m$^{-2}$ K$^{-1}$)}}
%\newglossaryentry{ra}{name={$r_{a}$},symbol={\ensuremath{r_{a}}},description={resistance from urban canopy to the atmosphere (s m$^{-1}$)}}
%\newglossaryentry{zu}{name={$z_{u}$},symbol={\ensuremath{z_{u}}},description={BoM wind speed measurement height (typically 10 m)}}
%\newglossaryentry{d}{name={$d$},symbol={\ensuremath{d}},description={zero plane displacement height (2/3 height of roughness elements, m)}}  
%\newglossaryentry{z0}{name={$z_{0}$},symbol={\ensuremath{z_{0}}},description={roughness length (0.1 $\times$ height of roughness elements)}}  
%
%\newglossaryentry{Hca}{name={$H_{ca}$},symbol={\ensuremath{H_{ca}}},description={conductance from urban canopy to the atmosphere (m s$^{-1}$)}}
%\newglossaryentry{Hcs}{name={$H_{cs}$},symbol={\ensuremath{H_{cs}}},description={conductance from surface to canopy (m s$^{-1}$)}}
%\newglossaryentry{rs}{name={$r_{s}$},symbol={\ensuremath{r_{s}}},description={resistance from surface to canopy (s m$^{-1}$)}}
%\newglossaryentry{Ucan}{name={$U_{can}$},symbol={\ensuremath{U_{can}}},description={wind speed in canyon (m s$^{-1}$)}}
%\newglossaryentry{Utop}{name={$U_{top}$},symbol={\ensuremath{U_{top}}},description={wind speed at the top of the canyon (m s$^{-1}$)}}
%  
%\newglossaryentry{aa}{name={$a$},symbol={\ensuremath{a}},description={a}}
%%hv = bulk transfer coefficient for moisture (1.4×10$^{-3}$) 
%%ra = resistance from urban canopy to the atmosphere (s m$^{-1}$)	
%%βK = amount of shortwave radiation immediately absorbed by the water layer (set to 0.45)
%%η = extinction coefficient
%%κwtr = eddy diffusivity of water (m$^{2}$ s$^{-1}$)
%
%

%\nomenclature{$T_{mrt}$}{mean radiant temperature (\SI{}{\degreeCelsius})}  
%\nomenclature{$UTCI$}{universal thermal climate index}


\end{document}

\endinput
%%
%% End of file `elsarticle-template-harv.tex'.
